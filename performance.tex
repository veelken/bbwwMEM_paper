\section{Performance}
\label{sec:performance}

The performance of the algorithm is studied in terms of the separation between the $\dihiggs$ signal and the $\ttbar$ background 
that is achieved by the likelihood ratio $P$ given by Eq.~(\ref{eq:memLR}),
using Monte Carlo (MC) simulated samples of 
$\dihiggs \to \Pbottom\APbottom \, \PW^{+}\PW^{-} \to \Pbottom\APbottom \, \ellPlusnu\ellMinusnu$ signal 
and $\ttbar \to \Pbottom\PW^{+} \, \APbottom\PW^{-} \to \Pbottom\ellPlusnu \, \APbottom\ellMinusnu$ background events.
The signal and background samples are generated \st{at LO accuracy with the program MadGraphaMCatNLO 2.2.2 CV: CHECK MADGRAPH VERSION USED TO PRODUCE MC SAMPLES,}
taking the $\PHiggs$ boson mass to be $m_{\PHiggs} = 125$~\GeV and a top quark mass to be $m_{\Ptop} = XXX.X$~\GeV[CV: CHECK TOP QUARK MASS USED TO PRODUCE MC SAMPLES]. \textcolor{red}{We already described the MC}
We have checked that the small differences between the masses used to generate the MC samples
and the mass values used in the reconstruction ($m_{\PHiggs} = 125.1$~\GeV and $m_{\Ptop} = 173.0$~\GeV, \cf Sections~\ref{sec:mem_signal} and~\ref{sec:mem_background})
has negligible effect.
All events are generated for proton-proton collisions at $\sqrt{s} = 13$~\TeV center-of-mass energy.
The NNPDF3.1 set of parton distribution functions have been used~\cite{NNPDF1,NNPDF2,NNPDF3}.
Parton shower and hadronization processes are modeled using the generator PYTHIA with the tune CP5~\cite{PYTHIA_CP5tune_CMS}.
Events containing $\PW^{+} \to \Pgt^{+}\Pnu_{\Pgt} \to \Plepton^{+}\Pnu_{\Plepton}\APnu_{\Pgt}\Pnu_{\Pgt}$ 
or $\PW^{-} \to \Pgt^{-}\APnu_{\Pgt} \to \Plepton^{-}\APnu_{\Plepton}\Pnu_{\Pgt}\APnu_{\Pgt}$ decays \st{,
where $\Plepton$ refers to an electron or muon,} are excluded from the performance study. \st{that we present in the following.}

We first study the events on MC truth level.
The experimental resolution on the energy of $\Pbottom$-jets and on the hadronic recoil still enters via the choice of the TF $W(E|\Ehat)$ in Eq.~(\ref{eq:TF_b})
and of the matrix $V$ in Eq.~(\ref{eq:TF_hadRecoil}).
Our choice of the TF is motivated by the resolutions achieved by the ATLAS and CMS experiments at the LHC~\cite{Aaboud:2017aca,PRF-14-001,Aaboud:2018tkc,JME-17-001}.
For the function $W(E|\Ehat)$, that models the energy resolution of $\Pbottom$-jets,
we choose a normal distribution:
\begin{equation}
W(E|\Ehat) = \frac{1}{\sqrt{2 \pi \sigma_{\Pbottom}^{2}}} \, e^{-\frac{(E - \Ehat)^{2}}{2 \, \sigma_{\Pbottom}^{2}}} \, ,
\label{eq:resolution_b}
\end{equation}
with standard deviation $\sigma_{\Pbottom} = 100\% \cdot \sqrt{\Ehat}$.
Concerning the resolution on the hadronic recoil,
we assume that the components $\pX^{\rho}$ and $\pY^{\rho}$ are each measured with a resolution of $\sigma_{\rho} = 10$~\GeV:
\begin{equation}
V = \left( \begin{array}{cc} \sigma_{\rho}^{2} & 0 \\ 0 & \sigma_{\rho}^{2} \end{array} \right) \, .
\label{eq:resolution_rho}
\end{equation}
Distributions of the likelihood ratio $P$ for simulated $\dihiggs$ signal and the $\ttbar$ background events is shown in Fig.~\ref{fig:memLR_and_ROC_unsmeared}.
Signal events are characterized by high values of $P$, while background events typically have low values.
The figure also shows the "receiver-operating-characteristic" (ROC) curve~\cite{ROCcurve} corresponding to the distribution in $P$.
The ROC curve is obtained by varying the threshold of a cut on the likelihood ratio $P$
and plotting the fractions of $\dihiggs$ signal and $\ttbar$ background events passing the cut.
It quantifies the separation between signal and background.
For a signal efficiency of $35\%$, the $\ttbar$ background is reduced by three orders of magnitude.

\begin{figure}
\setlength{\unitlength}{1mm}
\begin{center}
\begin{picture}(160,54)(0,0)
\put(0.0, 0.0){\mbox{\includegraphics*[height=54mm]
 {plots/hh_bbwwMEM_dilepton_signal_vs_background_memLR_unsmeared.pdf}}}
\put(81.0, 0.0){\mbox{\includegraphics*[height=54mm]
 {plots/hh_bbwwMEM_dilepton_ROC_unsmeared.pdf}}}
\end{picture}
\end{center}
\caption{
  Left: Distribution in the likelihood ratio $P$ given by Eq.~(\ref{eq:memLR}) for simulated $\dihiggs$ signal and $\ttbar$ background events.
  Right: ROC curve obtained by varying the cut threshold on $P$ for the distribution shown on the left
  The events are studied on MC truth level.
}
\label{fig:memLR_and_ROC_unsmeared}
\end{figure}

The effect of experimental resolutions on the distributions in $P$ and on the ROC curve is estimated
by randomly sampling the energy $E$ of $\Pbottom$-jets and the components $\pX^{\rho}$ and $\pY^{\rho}$ of the hadronic recoil from the corresponding TF,
given by Eqs.~(\ref{eq:TF_b}) and~(\ref{eq:TF_hadRecoil}), and recomputing the likelihood ratio $P$.
The parameters of the TF are taken from Eqs.~(\ref{eq:resolution_b}) and~(\ref{eq:resolution_rho}).
The resulting distributions in $P$ are shown in Fig.~\ref{fig:memLR_smeared},
compared to the distributions obtained for the case that the likelihood ratio $P$ is computed on MC truth level.
The effect on the separation between the $\dihiggs$ signal and the $\ttbar$ background is shown in Fig.~\ref{fig:ROC_smeared}.
As can be seen in the figure, the experimental resolutions have the effect of increasing the background rate by up to $20\%$ for a given signal efficiency,
compared to the case that the $\Pbottom$-jet energy and hadronic recoil can be measured with negligible experimental resolution.
Most of the effect is due to the resolution on the hadronic recoil,
while the resolution on the energy of $\Pbottom$-jets has merely a small effect.

\begin{figure}
\setlength{\unitlength}{1mm}
\begin{center}
\begin{picture}(160,78)(0,0)
\put(0.0, 0.0){\mbox{\includegraphics*[height=78mm]
 {plots/hh_bbwwMEM_dilepton_effectOfSmearing_memLR_signal.pdf}}}
\put(81.0, 0.0){\mbox{\includegraphics*[height=78mm]
 {plots/hh_bbwwMEM_dilepton_effectOfSmearing_memLR_background.pdf}}}
\end{picture}
\end{center}
\caption{
  Effect of the experimental resolutions on the energy of $\Pbottom$-jets (``$E_{\Pbottom}$ smearing'') and on the hadronic recoil (``$\rho$ smearing'') 
  on the distribution in the likelihood ratio $P$ for simulated $\dihiggs$ signal (left) and $\ttbar$ background (right) events.
}
\label{fig:memLR_smeared}
\end{figure}

\begin{figure}
\setlength{\unitlength}{1mm}
\begin{center}
\begin{picture}(160,78)(0,0)
\put(40.5, 0.0){\mbox{\includegraphics*[height=78mm]
 {plots/hh_bbwwMEM_dilepton_effectOfSmearing_ROC.pdf}}}
\end{picture}
\end{center}
\caption{
  Effect of the experimental resolutions on the energy of $\Pbottom$-jets (``$E_{\Pbottom}$ smearing'') and on the hadronic recoil (``$\rho$ smearing'') 
  on the separation between the $\dihiggs$ signal and the $\ttbar$ background,
  quantified by the ROC curve, based on the likelihood ratio $P$. 
}
\label{fig:ROC_smeared}
\end{figure}

Besides the experimental resolution on the $\Pbottom$-jet energy and on the hadronic recoil,
reconstruction effects may degrade the separation between the $\dihiggs$ signal and the $\ttbar$ background
in case one of the two $\Pbottom$-jets that are produced in the decays of the $\PHiggs$ boson or top quark pair
fails to get reconstructed,
and a light quark or gluon jet produced by either ISR or FSR, from the underlying event, 
or from additional proton-proton interactions (pileup) occurring in the same bunch-crossing as the hard-scattering interaction,
gets misidentified as $\Pbottom$-jet.
The efficiency to identify $\Pbottom$-jet typically amounts to $60$-$70\%$ at the ATLAS and CMS experiments,
for a misidentification rate for light quarks and gluon jets of order $1\%$~\cite{Aad:2015ydr,BTV-16-002}.

We simulate the effect that one of the true $\Pbottom$-jets fails to get reconstructed and a light quark or gluon jets gets misidentified as $\Pbottom$-jet,
by replacing $10\%$ of true $\Pbottom$-jets by randomly selected light quark or gluon jets produced by either ISR or FSR, or from the underlying event 
(our simulated samples of signal and background events do not include pileup). 
\textcolor{blue}{We then compare the response of MEM only on events that only have genuine $\Pbottom$-jets and only on events that have one or more induced fakes.}

\st{The distributions in $P$ obtained in case one of the two $\Pbottom$-jet candidates taken as input to the MEM 
is due to the misidentification of a light quark or gluon jet 
are compared to the distributions obtained for the case that both $\Pbottom$-jet candidates are genuine $\Pbottom$-jets in Fig. }
In Fig.~\ref{fig:memLR_fakeBJet} we see the distributions in $P$ obtained in case one of the two $\Pbottom$-jet candidates taken as input to the MEM 
is  misidentified and the same distribution obtained for the case that both $\Pbottom$-jet candidates are genuine $\Pbottom$-jets. 
\textcolor{red}{X: I tried to rephrase the previous sentence in a more direct way, I  had to read 3 times to understand the original. But, if you are comparing events that are genuine with events that have a certified fake, why did you did a sample }
The distribution in $P$ obtained for signal (background) events is seen to become significantly more background-like (signal-like) 
in case a light quark or gluon jet is misidentified as $\Pbottom$-jet candidate.
The corresponding ROC curve is shown in Fig.~\ref{fig:ROC_fakeBJet}.
The separation between the $\dihiggs$ signal and the $\ttbar$ background is affected significantly.
In case one of the two $\Pbottom$-jet candidates taken as input to the MEM is due to the misidentification of a light quark or gluon jet,
the rate of $\ttbar$ background obtained for a signal efficiency of $35\%$ increases by about two orders of magnitude 
compared to the case that both $\Pbottom$-jet candidates are genuine $\Pbottom$-jets.

Compared to the effect of the experimental resolution on the $\Pbottom$-jet energy and on the hadronic recoil,
the loss in signal-to-background separation arising from the misidentification of a light quark or gluon jet as $\Pbottom$-jet candidate
is much more severe. 
\textcolor{red}{X: I pulled this paragraph up.} 
We expect \textcolor{blue}{however} that the $\Pbottom$-jet candidate taken as input to the MEM will be a genuine $\Pbottom$-jet for the vast majority of signal and background events
selected in the $\Pbottom\Pbottom\PW\PW^{*}$ channel at the LHC.
The $\Pbottom$-tagging algorithms employed by the ATLAS and CMS collaborations provide a continuous discriminant, 
and we expect that sorting the $\Pbottom$-jet candidates by decreasing discriminant value and taking the $\Pbottom$-jet with the highest discriminant value as input to the MEM
will yield a genuine $\Pbottom$-jet in the vast majority of cases. \textcolor{red}{X: can we quote a number referent to the last publications? or an estimation, like "~5\% of the cases"?} 


\begin{figure}
\setlength{\unitlength}{1mm}
\begin{center}
\begin{picture}(160,54)(0,0)
\put(0.0, 0.0){\mbox{\includegraphics*[height=54mm]
 {plots/hh_bbwwMEM_dilepton_effectOfFakes_2histograms_memLR_signal.pdf}}}
\put(81.0, 0.0){\mbox{\includegraphics*[height=54mm]
 {plots/hh_bbwwMEM_dilepton_effectOfFakes_2histograms_memLR_background.pdf}}}
\end{picture}
\end{center}
\caption{
  Effect of misidentifying a light quark or gluon jet produced by either ISR or FSR, or from the underlying event (``fake'') as $\Pbottom$-jet candidate
  on the distribution in the likelihood ratio $P$ for simulated $\dihiggs$ signal (left) and $\ttbar$ background (right) events.
}
\label{fig:memLR_fakeBJet}
\end{figure}

\begin{figure}
\setlength{\unitlength}{1mm}
\begin{center}
\begin{picture}(160,54)(0,0)
\put(40.5, 0.0){\mbox{\includegraphics*[height=54mm]
 {plots/hh_bbwwMEM_dilepton_effectOfFakes_2graphs_ROC.pdf}}}
\end{picture}
\end{center}
\caption{
  Effect of misidentifying a light quark or gluon jet produced by either ISR or FSR, or from the underlying event (``fake'') as $\Pbottom$-jet candidate
  on the separation between the $\dihiggs$ signal and the $\ttbar$ background,
  quantified by the ROC curve, based on the likelihood ratio $P$.
}
\label{fig:ROC_fakeBJet}
\end{figure}

The magnitude of the effect on events with one or more misidentified $\Pbottom$-jets anyway warrants further investigations concerning its origin and the development of approaches to mitigate the effect.
In Fig.~\ref{fig:probS_and_probB_fakeBJet} we show the distributions in the probability densities $w_{0}$ and $w_{1}$,
which quantify the level of compatibility with the signal and background hypotheses,
for signal and background events in which one of the two $\Pbottom$-jet candidates taken as input to the MEM is due to the misidentification of a light quark or gluon jet,
and events in which both $\Pbottom$-jet candidates are genuine $\Pbottom$-jets.
We note that the distributions in the ``correct'' probability density 
($w_{0}$ for signal, $w_{1}$ for background events)
is more susceptible to the misidentification of light quark and gluon jets compared to the ``incorrect'' probability density 
($w_{0}$ for background, $w_{1}$ for signal events).
The distribution in $w_{0}$ for signal events is affected the most.
The large magnitude of the effect on $w_{0}$ is explained by the presence of a BW propagator in the ME $\mathcal{M}_{0}$ for the signal hypothesis,
which enforces that the mass of the pair of $\Pbottom$-jet candidates equals $m_{\PHiggs}$.
The $\delta$-function introduced into the integrand of Eq.~(\ref{eq:mem3}) via the transformations described in Section~\ref{sec:appendix_bEn_Hbb} of the appendix
guarantees the compliance with the $\PHiggs$ mass constraint,
albeit at the cost of introducing large ``pulls'' in the TF $W(E|\Ehat)$ for the $\Pbottom$-jet energy,
which diminish the value of the integrand in case one of the two $\Pbottom$-jet candidates taken as input to the MEM 
is due to the misidentification of a light quark or gluon jet.

\begin{figure}
\setlength{\unitlength}{1mm}
\begin{center}
\begin{picture}(160,116)(0,0)
\put(0.0, 62.0){\mbox{\includegraphics*[height=54mm]
 {plots/hh_bbwwMEM_dilepton_effectOfFakes_2histograms_probS_signal.pdf}}}
\put(81.0, 62.0){\mbox{\includegraphics*[height=54mm]
 {plots/hh_bbwwMEM_dilepton_effectOfFakes_2histograms_probB_signal.pdf}}}
\put(0.0, 0.0){\mbox{\includegraphics*[height=54mm]
 {plots/hh_bbwwMEM_dilepton_effectOfFakes_2histograms_probS_background.pdf}}}
\put(81.0, 0.0){\mbox{\includegraphics*[height=54mm]
 {plots/hh_bbwwMEM_dilepton_effectOfFakes_2histograms_probB_background.pdf}}}
\end{picture}
\end{center}
\caption{
  Effect of misidentifying a light quark or gluon jet produced by either ISR or FSR, or from the underlying event (``fake'') as $\Pbottom$-jet candidate
  on the distribution in the probability densities $w_{0}$ (left) and $w_{1}$ (right)
  for simulated $\dihiggs$ signal (top) and $\ttbar$ background (bottom) events.
}
\label{fig:probS_and_probB_fakeBJet}
\end{figure}

The degradation in signal-to-background separation can be mitigated 
by marginalizing the expressions for the probability densities $w_{0}$ and $w_{1}$ in Eqs.~(\ref{eq:mem_signal}) and~(\ref{eq:mem_background}),
assuming that one of the two true $\Pbottom$-jets failed to get reconstructed.
Marginalization means to integrate the RHS of these expressions over either the variables $E_{\Pbottom}$, $\theta_{\Pbottom}$, and $\phi_{\Pbottom}$ 
or over the variables $E_{\APbottom}$, $\theta_{\APbottom}$, and $\phi_{\APbottom}$,
corresponding to the reconstructed energy, polar, and azimuthal angle of either the $\Pbottom$ or $\APbottom$ quark.
The distributions in the marginalized likelihood ratio $P_{\textrm{m}}$ obtained for the $\dihiggs$ signal and for the $\ttbar$ background are shown in Fig.~\ref{fig:memLR_missingBJet},
and the corresponding ROC curve is shown in Fig.~\ref{fig:ROC_missingBJet}.
The distributions and ROC curve are shown separately for the case that the remaining (non-marginalized) $\Pbottom$-jet candidate is a genuine $\Pbottom$-jet
and the case that it is due to the misidentification of a light quark or gluon jet.
The marginalization reduces the amount of kinematic information that is utilized for separating the signal from the background,
resulting in an increased overlap in the distributions in $P_{\textrm{m}}$ for the $\dihiggs$ signal and for the $\ttbar$ background,
but also makes the distributions less susceptible to the non-reconstruction of one of the $\Pbottom$-jets.


\begin{figure}
\setlength{\unitlength}{1mm}
\begin{center}
\begin{picture}(160,54)(0,0)
\put(0.0, 0.0){\mbox{\includegraphics*[height=54mm]
 {plots/hh_bbwwMEM_dilepton_effectOfFakes_memLR_missingBJet_signal.pdf}}}
\put(81.0, 0.0){\mbox{\includegraphics*[height=54mm]
 {plots/hh_bbwwMEM_dilepton_effectOfFakes_memLR_missingBJet_background.pdf}}}
\end{picture}
\end{center}
\caption{
  Effect of misidentifying a light quark or gluon jet produced by either ISR or FSR, or from the underlying event (``fake'') as $\Pbottom$-jet candidate
  on the distribution in the marginalized likelihood ratio $P_{\textrm{m}}$ for simulated $\dihiggs$ signal (left) and $\ttbar$ background (right) events.
}
\label{fig:memLR_missingBJet}
\end{figure}

\begin{figure}
\setlength{\unitlength}{1mm}
\begin{center}
\begin{picture}(160,54)(0,0)
\put(40.5, 0.0){\mbox{\includegraphics*[height=54mm]
 {plots/hh_bbwwMEM_dilepton_effectOfFakes_ROC_missingBJet.pdf}}}
\end{picture}
\end{center}
\caption{
  Effect of misidentifying a light quark or gluon jet produced by either ISR or FSR, or from the underlying event (``fake'') as $\Pbottom$-jet candidate
  on the separation between the $\dihiggs$ signal and the $\ttbar$ background,
  quantified by the ROC curve, based on the marginalized likelihood ratio $P_{\textrm{m}}$.
}
\label{fig:ROC_missingBJet}
\end{figure}

\section{Usage and challenges}

We expect that searches for non-resonant $\dihiggs$ production in the channel $\Pbottom\Pbottom\PW\PW^{*}$ at the LHC
will benefit from using a multivariate approach, for example based on a BDT or Neural Network (NN)~\cite{TensorFlow,Keras},
using the likelihood ratio $P$ given by Eq.~(\ref{eq:memLR}) and the marginalized one $P_{\textrm{m}}$ as input,
together with other observables such as the number of $\Pbottom$-jet candidates and their $\Pbottom$-tagging discriminant values, 
the mass of the two jets with the highest $\Pbottom$-tagging discriminant, and other kinematic observables.
Other, subdominant, backgrounds can be included in the training of the BDT or NN
to further improve the separation of the $\dihiggs$ signal from backgrounds.

One remaining issue in practical applications of the MEM may be the computing time requirements.
Experimental analyses will usually need to evaluate the integrals in Eqs.~(\ref{eq:mem_signal}) and~(\ref{eq:mem_background})
multiple times for each event \st{and benchmark scenario} 
in order to assess the effect of systematic uncertainties.
Taken together with the large cross section for $\ttbar$ production at the LHC,
the integrals in Eqs.~(\ref{eq:mem_signal}) and~(\ref{eq:mem_background}) will need to be computed in the order of $100$ million times.
Even with several thousands of computing jobs running in parallel,
as it is nowadays commonplace for experimental analyses performed at the LHC,
the computation still requires a few weeks of nonstop computing time.
Several possibilities to speed up the numeric integrations, which take most of the computing time in practical applications of the MEM,
have been explored in the literature.
One alternative is to use vector integrands to evaluate the likelihood ratio for all systematic uncertainties simultaneously~\cite{CUBA},
taking advantage of the fact that the systematic uncertainties typically constitute small changes with respect to the nominal value.
Another alternative is to take advantage of the parallelizability of multidimensional integration and perform the integration on graphics processing units (GPUs).
Speedup factors of order $100$, compared to using a single core of a general-purpose central processing unit (CPU) 
such as the $2.30$~GHz Intel\TReg~Xeon\TReg~E5-2695V3 processor that we used for the studies presented in this paper,
are reported in the literature for performing numeric integrations on GPUs~\cite{Hagiwara:2009aq,Hagiwara:2009cy,Kanzaki:2010ym,Hagiwara:2013oka,Schouten:2014yza,Grasseau:2015vfa}.

\section{Extension for application on non-resonant BSM HH production}

So far, we have focused on non-resonant $\dihiggs$ production with SM-like event kinematics.
\textcolor{blue}{As already mentioned,} in case the $\dihiggs$ production rate is enhanced compared to the SM cross section, the scenario probed by the present searches for $\dihiggs$ production at the LHC,
\st{it is reasonable to expect that the} contributions of BSM physics, which enhance the cross section, also alter the event kinematics.
In the context of an EFT approach to probe the existence of such BSM physics,
we recommend to compute the likelihood ratios $P$ and $P_{\textrm{m}}$ for the SM case and each of the $12$ benchmark scenarios detailed in Refs.~\cite{Carvalho:2015ttv,Carvalho:2016rys},
and run a parametrized training of the BDT or NN for these shape benchmark scenarios. \textcolor{red}{If we want to explain that I need to spend some time explaining the method of finding the closest  shape benchmark.... it goes as other 1/2 page paragraph -- the sentence bellow makes no sense, and I commented we already said that the model allows parameters.} 
%\textcolor{blue}{Using the same model [cite] we can also }  ME $\vert \mathcal{M}_{0} \vert^{2}$ \st{obtained from the program MadGraph\_aMCatNLO 2.2.2~\cite{MadGraph_aMCatNLO}[CV: CHECK MADGRAPH VERSION USED TO GENERATE ME]}
%allows to set the two parameters $\kappa_{\lambda}$ and $\kappa_{\Ptop}$ and the coefficients $c_{2}$, $c_{\Pgluon}$, and $c_{2\Pgluon}$
%to the values given in Table~1 of Ref.~\cite{Carvalho:2015ttv} for the SM case and each of the $12$ benchmark scenarios.
