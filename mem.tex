\section{The Matrix element method}
\label{sec:mem}

The MEM computes probability densities $w_{i}(\vecy|\vecyhat)$
for obtaining the measured observables $\vecy$, assuming the event has been produced by the process $i$.
The probability densities $w_{i}(\vecy|\vecyhat)$ are interpreted as quantifying the compatibility of the measured observables $\vecy$
with the signal ($i=0$) and background ($i=1$) hypothesis.
In the analysis of $\PHiggs\PHiggs$ production in the decay channel 
$\PHiggs\PHiggs \to \Pbottom\APbottom \, \PW^{+}\PW^{-} \to \Pbottom\APbottom \, \ellPlusnu\ellMinusnu$
the observables $\vecy$ refer to 
the measured momenta of the two $\Pbottom$-jets, the two charged leptons, and the measured missing transverse momentum ($\vecMET$) in the event.
We use symbols with a hat to denote the true values of energies and momenta.
Bold letters denote vector quantities.
The vector $\vecyhat$ denotes the true values of the $\Pbottom$-jet and charged lepton momenta and the true values of the momenta of the two neutrinos produced in the $\PW$ boson decays.

\textcolor{blue}{As already mentioned} \st{Note that} $\PHiggs\PHiggs \to \Pbottom\APbottom \, \PW^{+}\PW^{-} \to \Pbottom\APbottom \, \ellPlusnu\ellMinusnu$ signal events
contain the same number of $\Pbottom$-jets, charged leptons, and neutrinos as the dominant background,
arising from top quark pair production ($\ttbar \to \Pbottom\PW^{+} \, \APbottom\PW^{-} \to \Pbottom\ellPlusnu \, \APbottom\ellMinusnu$). \st{,
in this channel.}
The separation of the $\PHiggs\PHiggs$ signal from the irreducible $\ttbar$ background is based on the difference in event kinematics,
causing the probability densities $w_{0}(\vecy|\vecyhat)$ ($w_{1}(\vecy|\vecyhat)$) to be in general higher when evaluated on signal (background) events
and lower when evaluated on background (signal) events.
Given the probability densities $w_{0}(\vecy|\vecyhat)$ and $w_{1}(\vecy|\vecyhat)$ for the signal and background hypotheses,
the Neyman–Pearson lemma~\cite{Neyman:1937uhy} postulates that the likelihood ratio:
\begin{equation}
P(\vecy) = \frac{w_{0}(\vecy|\vecyhat)}{w_{0}(\vecy|\vecyhat) + w_{1}(\vecy|\vecyhat)}
\label{eq:memLR}
\end{equation}
\st{then} provides the optimal separation of the $\PHiggs\PHiggs$ signal from the irreducible $\ttbar$ background.

Different nomenclatures and conventions for the MEM exist in the literature.
In this note, we follow the nomenclature and conventions introduced in Ref.~\cite{Volobouev:2011vb}.
The probability densities $w_{i}(\vecy|\vecyhat)$ are given by the integral:
\begin{align}
w_{i}(\vecy|\vecyhat) = & \frac{\Omega(\vecy)}{\sigma \, \mathcal{A}} \, \int \, d\xhat_{a} \, d\xhat_{b} \,
  d\Phi_{n} \, \frac{f(\xhat_{a}) \, f(\xhat_{b})}{2 \, \xhat_{a} \, \xhat_{b} \, s} \, (2\pi)^{4} \,
  \delta\left( \xhat_{a} \, \Ehat_{a} + \xhat_{b} \, \Ehat_{b} - \sum_{i}^{n} \Ehat_{(i)} \right) \nonumber \\
 & \quad \cdot \, \delta^{3}( \xhat_{a} \, \vecphat^{a} + \xhat_{b} \, \vecphat^{b} - \sum_{i}^{n} \vecphat^{(i)}) \, 
  \vert \mathcal{M}_{i}(\vecyhat) \vert^{2} \, W(\vecy|\vecyhat) \, \epsilon(\vecyhat) \, .
\label{eq:mem1}
\end{align}
The symbol $\vert \mathcal{M}_{i}(\vecyhat) \vert^{2}$ denotes the squared modulus of the matrix element (ME) for either the signal ($i=0$) or for the background ($i=1$) hypothesis.
We use ME generated at LO accuracy with the program MadGraph\_aMCatNLO 2.2.2~\cite{MadGraph_aMCatNLO} for the signal as well as for the background hypothesis[CV: CHECK MADGRAPH VERSION USED TO GENERATE ME]. \textcolor{red}{X: as referee I would ask why so old version, this is to be compatible with the model that is based on form factors -- that BTW we need to cite. We use this old model because we could not use the loop induced by some practical reason that I do not remember. I would do a chapter "MC modelling", where we collect the MC info instead of having it popping up eventually.}

The symbols $\Ehat_{a}$ and $\Ehat_{b}$ ($\vecphat^{a}$ and $\vecphat^{b}$) denote the energies (momenta) of the two colliding protons,
$\sqrt{s}$ their center-of-mass energy,
$\xhat_{a}$ and $\xhat_{b}$ the Bjorken scaling variables~\cite{Bjorkenx},
and $f(\xhat_{a})$ and $f(\xhat_{b})$ the corresponding parton distribution functions (PDF).
We use the MSTW 2008 LO PDF set~\cite{MSTW} to evaluate $f(\xhat_{a})$ and $f(\xhat_{b})$.
We denote by $n$ the number of particles in the final state,
by $\vecp^{(i)}$ ($\vecphat^{(i)}$) the measured (true) momentum of the $i$-th final state particle. 
The $\delta$-functions $\delta( \xhat_{a} \, \Ehat_{a} + \xhat_{b} \, \Ehat_{b} - \sum_{i}^{n} \Ehat_{(i)})$
and $\delta^{3}( \xhat_{a} \, \vecphat^{a} + \xhat_{b} \, \vecphat^{b} - \sum_{i}^{n} \vecphat^{(i)})$ 
impose conservation of energy and momentum.

The functions $W(\vecy|\vecyhat)$ are referred to as ``transfer functions'' (TF) in the literature.
They represent the probability density to observe the measured values $\vecy$, given the true values $\vecyhat$.
The vector $\vecyhat$ corresponds to a point $\vecphat$ in the $n$-particle PS.
The function $\Omega(\vecy)$ is referred to as ``indicator function'' in the literature~\cite{Fiedler:2010sg,Volobouev:2011vb}.
It attains the value $1$ in case the event represented by the measured observables $\vecy$ passes the event selection criteria and otherwise attains the value $0$.
The efficiency for an event originating at the PS point
$\vecyhat$ to pass the event selection, \ie to end up with measured
observables $\vecy$ for which $\Omega(\vecy) = 1$,
is denoted by $\epsilon(\vecy)$. 
Finally, the symbol $\mathcal{A}$ denotes the acceptance of the event selection, 
that is, the percentage of events which pass the event selection criteria,
while $\sigma$ denotes the total (inclusive) cross section.
Division of the right-hand-side (RHS) of Eq.~(\ref{eq:mem1}) by the product $\sigma \, \mathcal{A}$
ensures that $w_{i}(\vecy|\vecyhat)$ has the correct normalization required for a probability density, 
\ie $\int \, d\vecy \, w_{i}(\vecy|\vecyhat) = 1$,
provided that the TF satisfy the normalization condition
$\int \, d\vecy \, \Omega(\vecy) \, W(\vecy|\vecyhat) = 1$
for every $\vecy$.

The $d\Phi_{n} = \prod_{i}^{n} \, \frac{d^{3}\vecphat^{(i)}}{(2\pi)^{3} \, 2 \Ehat_{(i)}}$ 
represents the differential $n$-particle phase space (PS) element.
For the $\PHiggs\PHiggs$ signal 
as well as for the $\ttbar$ background,
$n=6$.  \st{as the final state in both cases contains a pair of $\Pbottom$ and $\APbottom$ quarks, two charged leptons, and two neutrinos.} \textcolor{red}{X:I just changed the order of the presentation of pieces to not have to repeat again the counting.}
We express the PS element $d\Phi_{6}$ in terms of the energies $\Ehat_{(i)}$, the polar angles $\thetahat_{(i)}$, and the azimuthal angles $\phihat_{(i)}$ 
of the pair of $\Pbottom$ and $\APbottom$ quarks, of the two charged leptons, and of the two neutrinos:
\begin{eqnarray}
d\Phi_{6} 
 & = &\prod_{i}^{6} \, \frac{d^{3}\vecphat^{(i)}}{(2 \, \pi)^{3} \, 2 \, \Ehat_{(i)}} 
  = \frac{1}{2^{24} \, \pi^{18}} \, \prod_{i}^{6} \, 
\frac{d^{3}\vecphat^{(i)}}{\Ehat_{(i)}} \nonumber \\
 & = & \frac{1}{2^{24} \, \pi^{18}} \, \prod_{i}^{6} \, 
\frac{d\Ehat_{(i)} \, d\thetahat_{(i)} \, d\phihat_{(i)} \, |\vecphat^{(i)}| \, \Ehat_{(i)} \, \sin\thetahat_{(i)}}{\Ehat_{(i)}} \nonumber \\
 & = & \frac{1}{2^{24} \, \pi^{18}} \, \prod_{i}^{6} \, 
d\Ehat_{(i)} \, d\thetahat_{(i)} \, d\phihat_{(i)} \, \betahat_{(i)} \Ehat_{(i)} \, \sin\thetahat_{(i)} \, .
\label{eq:PS_inPolarCoordinates}
\end{eqnarray}
All energies $\Ehat_{(i)}$ as well as the angles $\thetahat_{(i)}$ and $\phihat_{(i)}$ refer to the laboratory (detector) frame.
The velocity $\betahat_{(i)}$ of particle $i$,
given by $\betahat_{(i)} \equiv \frac{|\vecphat^{(i)}|}{\Ehat_{(i)}}$,
has been used to simplify the expression for $d\Phi_{6}$ in the last step.
Note that the velocity $\betahat_{(i)}$ is a function of energy $\Ehat_{(i)}$ and hence cannot be treated as constant when evaluating the integral over $d\Ehat_{(i)}$.
Similarly, the magnitude of the momentum $|\vecphat^{(i)}|$ is a function of the energy $\Ehat_{(i)}$.
In the following, we use the identities $|\vecphat^{(i)}| = \sqrt{\Ehat_{(i)}^{2} - m_{(i)}^{2}}$ 
and $\betahat_{(i)} = \frac{\sqrt{\Ehat_{(i)}^{2} - m_{(i)}^{2}}}{\Ehat_{(i)}}$ to make the dependency on the energy $\Ehat_{(i)}$ explicit.

The form of Eq.~(\ref{eq:PS_inPolarCoordinates}) is useful, 
as it allows to trivially perform the integration over the 
angles $\thetahat_{(i)}$ and $\phihat_{(i)}$ for the pair of $\Pbottom$ and $\APbottom$ quarks and for the two charged leptons,
taking advantage of the fact that the directions of quarks (jets) and charged leptons are measured with negligible experimental resolution.
With the further assumption that also the energy of charged leptons can be measured with negligible experimental resolution,
the integration over $d\Ehat_{\Plepton^{+}}$ and $d\Ehat_{\Plepton^{-}}$ can be carried out trivially too.
We shall only consider events that pass the event selection criteria, \ie for which the indicator function $\Omega(\vecy)$ is equal to $1$.
For simplicity, we \st{moreover} neglect the effect of the efficiency $\epsilon(\vecyhat)$ and of the acceptance $\mathcal{A}$.
With these assumptions and upon inserting the expressions for the TF given by Eqs.~(\ref{eq:TF_ell}),~(\ref{eq:TF_b}), and~(\ref{eq:TF_f}) in the appendix
into Eq.~(\ref{eq:PS_inPolarCoordinates}), we obtain:
\begin{align}
w_{i}(\vecy|\vecyhat) 
 = & \frac{1}{2^{24} \, \pi^{18} \, \sigma \, E_{\Plepton^{+}} \, E_{\Plepton^{-}}} \, \int \, d\xhat_{a} \, d\xhat_{b} \,
\frac{d\Ehat_{\Pbottom}}{\Ehat_{\Pbottom}} \, \frac{d\Ehat_{\APbottom}}{\Ehat_{\APbottom}} \, \frac{d^{3}\vecphat_{\Pnu}}{\Ehat_{\Pnu}} \, \frac{d^{3}\vecphat_{\APnu}}{\Ehat_{\APnu}} \,
\frac{f(\xhat_{a}) \, f(\xhat_{b})}{2 \, \xhat_{a} \, \xhat_{b} \, s} \nonumber \\
 & \quad \cdot \, (2 \, \pi)^{4} \, \delta( \xhat_{a} \, \Ehat_{a} + \xhat_{b} \, \Ehat_{b} - \sum_{i}^{6} \Ehat_{(i)}) \, 
\delta^{3}( \xhat_{a} \, \vecphat^{a} + \xhat_{b} \, \vecphat^{b} - \sum_{i}^{6} \vecphat^{(i)}) \nonumber \\
 & \quad \cdot \, \vert \mathcal{M}_{i}(\vecphat) \vert^{2} \, \frac{\betahat_{\Pbottom} \, \Ehat_{\Pbottom}^{2}}{\beta_{\Pbottom} \, E_{\Pbottom}^{2}} \, W(E_{\Pbottom}|\Ehat_{\Pbottom}) \, 
\frac{\betahat_{\APbottom} \, \Ehat_{\APbottom}^{2}}{\beta_{\APbottom} \, E_{\APbottom}^{2}} \, W(E_{\APbottom}|\Ehat_{\APbottom}) \nonumber \\
 = & \frac{1}{2^{21} \, \pi^{14} \, \sigma \, E_{\Plepton^{+}} \, E_{\Plepton^{-}} \, E_{\Pbottom} \, E_{\APbottom}} \, \int \, d\xhat_{a} \, d\xhat_{b} \,
d\Ehat_{\Pbottom} \, d\Ehat_{\APbottom} \, \frac{d^{3}\vecphat_{\Pnu}}{\Ehat_{\Pnu}} \, \frac{d^{3}\vecphat_{\APnu}}{\Ehat_{\APnu}} \,
\frac{f(\xhat_{a}) \, f(\xhat_{b})}{\xhat_{a} \, \xhat_{b} \, s} \nonumber \\
 & \quad \cdot \, \delta( \xhat_{a} \, \Ehat_{a} + \xhat_{b} \, \Ehat_{b} - \sum_{i}^{6} \Ehat_{(i)}) \,
\delta^{3}( \xhat_{a} \, \vecphat^{a} + \xhat_{b} \, \vecphat^{b} - \sum_{i}^{6} \vecphat^{(i)}) \nonumber \\
 & \quad \cdot \, \vert \mathcal{M}_{i}(\vecphat) \vert^{2}\, \frac{\betahat_{\Pbottom} \, \Ehat_{\Pbottom}}{\beta_{\Pbottom} \, E_{\Pbottom}} \, W(E_{\Pbottom}|\Ehat_{\Pbottom}) \, 
\frac{\betahat_{\APbottom} \, \Ehat_{\APbottom}}{\beta_{\APbottom} \, E_{\APbottom}} \, W(E_{\APbottom}|\Ehat_{\APbottom}) \, .
\label{eq:mem2}
\end{align}
The terms $\frac{\betahat_{\Pbottom} \, \Ehat_{\Pbottom}}{\beta_{\Pbottom} \, E_{\Pbottom}}$ and $\frac{\betahat_{\APbottom} \, \Ehat_{\APbottom}}{\beta_{\APbottom} \, E_{\APbottom}}$ 
arise because the integration over the PS elements $d^{3}\vecphat$ of the $\Pbottom$ and $\APbottom$ quarks yields a factor $\betahat \, \Ehat^{2} \, \sin\thetahat$,
while the normalization of the TF yields a factor $\frac{1}{\beta \, E^{2} \, \sin\theta}$, \cf Eq.~(\ref{eq:TF_f}).
The terms $\sin\thetahat$ and $\frac{1}{\sin\theta}$ cancel, due to the presence of the $\delta$-function $\delta(\theta - \thetahat)$ in the integrand, \cf Eq.~(\ref{eq:TF_b}).
No similar terms arise for the charged leptons, as the TF for charged leptons demand $\betahat = \beta$, $\Ehat = E$, and $\thetahat = \theta$, \cf Eq.~(\ref{eq:TF_ell}).

Concerning the four-dimensional $\delta$-function 
$\delta( \xhat_{a} \, \Ehat_{a} + \xhat_{b} \, \Ehat_{b} - \sum_{i}^{6} \Ehat_{(i)}) \cdot \delta^{3}( \xhat_{a} \, \vecphat^{a} + \xhat_{b} \, \vecphat^{b} - \sum_{i}^{6} \vecphat^{(i)})$,
which ensures the conservation of energy and momentum,
we first eliminate the energy and longitudinal momentum components by solving for the Bjorken scaling variables $\xhat_{a}$ and $\xhat_{b}$.
This yields:
\begin{equation}
\xhat_{a} = \frac{1}{\sqrt{s}} \, \sum_{i}^{6} \Ehat_{(i)} + \pZhat^{(i)} \quad \mbox{ and } \quad
\xhat_{b} = \frac{1}{\sqrt{s}} \, \sum_{i}^{6} \Ehat_{(i)} - \pZhat^{(i)} \, .
\label{eq:Bjorkenx}
\end{equation}

For the purpose of eliminating the transverse momentum components of the four-dimensional $\delta$-function,
we follow the approach of Ref.~\cite{SVfitMEM}.
The approach is based on introducing the ``hadronic recoil'', denoted by the symbol $\rho$, as a means to account for QCD radiation,
which causes additional jets to be produced besides the two $\Pbottom$-jets originating from the decay of the $\PHiggs$ boson (in signal events) 
or from the decay of the two top quarks (in background events).
As detailed in Ref.~\cite{Alwall:2010cq}, significant amounts of QCD radiation, in particular initial-state radiation (ISR),
are a typical feature of most signal and background processes at the LHC.
The longitudinal momentum of the additional jets produced by QCD radiation alters the relations for $\xhat_{a}$ and $\xhat_{b}$ somewhat,
compared to the values given by Eq.~(\ref{eq:Bjorkenx}).
We expect the effect of QCD radiation on the energy and longitudinal momentum components to be small\st{, however,} and thus neglect it.
The effect on the transverse momentum balance is important, however,
as QCD radiation distorts the kinematic relations that would be expected to hold in the absence of such radiation.
As a consequence, the $\delta$-functions that ensure the conservation of momentum in the transverse plane need to be modified. 
Their modified form reads: 
$\delta( \pXhat^{\rho} + \sum_{i}^{6} \pXhat^{(i)})$ and $\delta( \pYhat^{\rho} + \sum_{i}^{6} \pYhat^{(i)})$,
where $\pXhat^{\rho}$ ($\pYhat^{\rho}$) denotes the true value of the momentum of the hadronic recoil $\rho$ in $x$ ($y$) direction.
The $\delta$-functions hold on the level of measured and on the level of true values.
For the measured values, they imply the relations:
\begin{equation*}
\pX^{\rho} = - \left( \pX^{\Pbottom} + \pX^{\APbottom} + \pX^{\Plepton^{+}} + \pX^{\Plepton^{-}} + \METx \right) \, \mbox{ and } \,
\pY^{\rho} = - \left( \pY^{\Pbottom} + \pY^{\APbottom} + \pY^{\Plepton^{+}} + \pY^{\Plepton^{-}} + \METy \right) \, ,
\end{equation*}
while the corresponding relations for the true momenta read:
\begin{equation}
\pXhat^{\rho} = - \left( \pXhat^{\Pbottom} + \pXhat^{\APbottom} + \pXhat^{\Plepton^{+}} + \pXhat^{\Pnu} + \pXhat^{\Plepton^{-}} + \pXhat^{\APnu} \right) \, \mbox{ and } \,
\pYhat^{\rho} = - \left( \pYhat^{\Pbottom} + \pYhat^{\APbottom} + \pYhat^{\Plepton^{+}} + \pYhat^{\Pnu} + \pYhat^{\Plepton^{-}} + \pYhat^{\APnu} \right) \, .
\label{eq:hadRecoil_true}
\end{equation}
The experimental resolution on $\pX^{\rho}$ and $\pY^{\rho}$ is modeled by introducing a TF for the hadronic recoil into the integrand of Eq.~(\ref{eq:mem2}).
We assume that the resolution on the transverse momentum components of $\rho$ follows a two-dimensional normal distribution:
\begin{equation}
W_{\rho}( \pX^{\rho},\pY^{\rho} | \pXhat^{\rho},\pYhat^{\rho} ) = 
 \frac{1}{2\pi \, \sqrt{\vert V \vert}} \, \exp \left( -\frac{1}{2}
 \left( \begin{array}{c} \pX^{\rho} - \pXhat^{\rho} \\ \pY^{\rho} - \pYhat^{\rho} \end{array} \right)^{T}
  \cdot V^{-1} \cdot
   \left( \begin{array}{c} \pX^{\rho} - \pXhat^{\rho} \\ \pY^{\rho} - \pYhat^{\rho} \end{array} \right)
 \right) \, ,
\label{eq:TF_hadRecoil}
\end{equation}
where the matrix $V$ quantifies the resolution on the hadronic recoil in the transverse plane.

The CMS collaboration computes the matrix $V$ event-by-event, using an algorithm referred to as ``$\MET$-significance'' algorithm~\cite{JME-10-009}. \textcolor{red}{in the context of the XX analysis??? Maybe it worths to mention that this is an adaptation of what is being used on CMS Htata on intro to make this comment here?} 
In analogy to the procedure detailed in Ref.~\cite{SVfitMEM}, we replace the $\delta$-functions 
$\delta( \pXhat^{\rho} + \sum_{i}^{6} \pXhat^{(i)})$ and $\delta( \pYhat^{\rho} + \sum_{i}^{6} \pYhat^{(i)})$
that ensure the momentum conservation in the transverse plane
by the TF for the hadronic recoil.

A remaining issue is that we use LO ME $\mathcal{M}_{i}(\vecphat)$ for the $\PHiggs\PHiggs$ signal and for the $\ttbar$ background in Eq.~(\ref{eq:mem2}).
The LO ME for the signal (background) requires that the $\PHiggs\PHiggs$ ($\ttbar$) system has zero $\pT$, 
a condition that (in the laboratory frame) only holds in case the hadronic recoil has zero $\pT$, 
which rarely holds at the LHC, due to the abundance of ISR (as previously discussed).
\textcolor{red}{Again I am confused ISR and FSR come from showering. LO means that the system has zero momenta on the HH(tt) CM. NLO is not only about ISR/FSR.} 
The issue that the LO ME is only well-defined for events with zero ISR
is resolved by evaluating the ME $\mathcal{M}_{i}(\vecyhat)$ in a frame in which the $\PHiggs\PHiggs$ ($\ttbar$) system has zero $\pT$, 
to which we refer as the zero-transverse-momentum (ZTM) frame.
The Lorentz transformation of the energy $\Ehat_{(i)}$ and momenta $\vecphat^{(i)}$ in Eq.~(\ref{eq:mem2})
from the laboratory to the ZTM frame is performed using the vector $(-\frac{\pX^{\rho}}{\pT^{\rho}},-\frac{\pY^{\rho}}{\pT^{\rho}},0)$ as the boost vector.

Eliminating the energy and longitudinal momentum components of the (modified) four-dimensional $\delta$-function 
$\delta( \xhat_{a} \, \Ehat_{a} + \xhat_{b} \, \Ehat_{b} - \sum_{i}^{6} \Ehat_{(i)}) \cdot \delta( \xhat_{a} \, \pZhat^{a} + \xhat_{b} \, \pZhat^{b} - \sum_{i}^{6} \pZhat^{(i)}) \cdot \delta( \pXhat^{\rho} + \sum_{i}^{6} \pXhat^{(i)}) \cdot \delta( \pYhat^{\rho} + \sum_{i}^{6} \pYhat^{(i)})$
by means of Eq.~(\ref{eq:Bjorkenx})
and replacing its transverse momentum components by the TF $W_{\rho}( \pX^{\rho},\pY^{\rho} | \pXhat^{\rho},\pYhat^{\rho} )$ for the hadronic recoil $\rho$,
the expression for the probability density $w_{i}(\vecy|\vecyhat)$ becomes:
\begin{align}
w_{i}(\vecy|\vecyhat) 
 = & \frac{1}{2^{21} \, \pi^{14} \, \sigma \, E_{\Plepton^{+}} \, E_{\Plepton^{-}} \, E_{\Pbottom} \, E_{\APbottom}} \, \int \, 
d\Ehat_{\Pbottom} \, d\Ehat_{\APbottom} \, \frac{d^{3}\vecphat_{\Pnu}}{\Ehat_{\Pnu}} \, \frac{d^{3}\vecphat_{\APnu}}{\Ehat_{\APnu}} \,
\frac{f(\xhat_{a}) \, f(\xhat_{b})}{\xhat_{a} \, \xhat_{b} \, s} \nonumber \\
 & \quad \cdot \, \vert \mathcal{M}_{i}(\vecphat) \vert^{2} \, 
\frac{\betahat_{\Pbottom} \, \Ehat_{\Pbottom}}{\beta_{\Pbottom} \, E_{\Pbottom}} \, W(E_{\Pbottom}|\Ehat_{\Pbottom}) \, 
\frac{\betahat_{\APbottom} \, \Ehat_{\APbottom}}{\beta_{\APbottom} \, E_{\APbottom}} \, W(E_{\APbottom}|\Ehat_{\APbottom}) \, W_{\rho}( \pX^{\rho},\pY^{\rho} | \pXhat^{\rho},\pYhat^{\rho} ) \nonumber \\
 = & \frac{1}{2^{21} \, \pi^{14} \, \sigma \, E_{\Plepton^{+}} \, E_{\Plepton^{-}} \, E_{\Pbottom} \, E_{\APbottom}} \, \int \, 
d\Ehat_{\Pbottom} \, d\Ehat_{\APbottom} \, d\Ehat_{\Pnu} \, d\thetahat_{\Pnu} \, d\phihat_{\Pnu} \, d\Ehat_{\APnu} \, d\thetahat_{\APnu} \, d\phihat_{\APnu} \nonumber \\
 & \quad \cdot \, \betahat_{\Pnu} \, \Ehat_{\Pnu} \, \sin\thetahat_{\Pnu} \, 
\betahat_{\APnu} \, \Ehat_{\APnu} \, \sin\thetahat_{\APnu} \, 
\frac{f(\xhat_{a}) \, f(\xhat_{b})}{\xhat_{a} \, \xhat_{b} \, s} \nonumber \\
 & \quad \cdot \, \vert \mathcal{M}_{i}(\vecphat) \vert^{2} \, 
\frac{\betahat_{\Pbottom} \, \Ehat_{\Pbottom}}{\beta_{\Pbottom} \, E_{\Pbottom}} \, W(E_{\Pbottom}|\Ehat_{\Pbottom}) \, 
\frac{\betahat_{\APbottom} \, \Ehat_{\APbottom}}{\beta_{\APbottom} \, E_{\APbottom}} \, W(E_{\APbottom}|\Ehat_{\APbottom}) \, W_{\rho}( \pX^{\rho},\pY^{\rho} | \pXhat^{\rho},\pYhat^{\rho} ) \, .
\label{eq:mem3}
\end{align}
The expression in Eq.~(\ref{eq:mem3}) concludes our discussion of the analytic transformations that we apply to $w_{i}(\vecy|\vecyhat)$
in case of the signal ($i=0$) as well as the background ($i=1$) hypothesis.

A few more transformations are necessary to handle the presence of Breit-Wigner (BW) propagators in the ME $\mathcal{M}_{i}(\vecphat)$ while performing the numeric integration.
The effect of the BW propagators is that only narrow slices in the $6$-particle PS yield sizeable contribution to the integral,
namely the regions where the $6$ final state particles \st{(the $\Pbottom$ and $\APbottom$ quarks, the two charged leptons, and the two neutrinos)}
satisfy certain mass constraints.
The mass constraints reflect the presence of (on-shell) $\PHiggs$ bosons, $\PW$ bosons, and top quarks in the decay chains
$\PHiggs\PHiggs \to \Pbottom\APbottom \, \PW^{+}\PW^{-} \to \Pbottom\APbottom \, \ellPlusnu\ellMinusnu$ and
$\ttbar \to \Pbottom\PW^{+} \, \APbottom\PW^{-} \to \Pbottom\ellPlusnu \, \APbottom\ellMinusnu$.
The presence of these narrow slices in the $6$-particle PS renders the numeric integration inefficient, unless the mass constraints are treated analytically. We use the narrow-width approximation~\cite{NWA} to treat the mass constraints and replace the BW propagators by $\delta$-functions,
which has the effect of restricting the numerical integration to the regions in PS where the mass constraints are satisfied
and the ME $\mathcal{M}_{i}(\vecphat)$ yields a sizeable contribution to the integral.
The analytic transformations specific to the signal hypothesis are detailed in Section~\ref{sec:mem_signal},
while those specific to the background hypothesis are presented in Section~\ref{sec:mem_background}.

The final numeric integration is performed using the VAMP algorithm~\cite{VAMP}, a variant of the popular VEGAS algorithm~\cite{VEGAS},
which has been optimized for the case of integrating multimodal functions, typically appearing in the integration of ME over regions in PS.
We use $2500$ ($25000$) evaluations of the integrand for the computation of the probability density $w_{0}(\vecy|\vecyhat)$ ($w_{1}(\vecy|\vecyhat)$) 
for the signal (background) hypothesis.
The number of evaluations has been chosen such that the computation of $w_{0}(\vecy|\vecyhat)$ and $w_{1}(\vecy|\vecyhat)$ take approximately the same time
and the computation of the likelihood ratio $P(\vecy)$ takes about one minute per event,
using a single core of a $2.30$~GHz Intel\TReg~Xeon\TReg~E5-2695V3 processor.


\subsection{Analytic transformations specific to the signal hypothesis}
\label{sec:mem_signal}

In $\PHiggs\PHiggs \to \Pbottom\APbottom \, \PW^{+}\PW^{-} \to \Pbottom\APbottom \, \ellPlusnu\ellMinusnu$ signal events,
either the $\PW^{+}$ or the $\PW^{-}$ boson (but not both) are on mass shell.
We denote the off-shell $\PW$ boson in signal events by an asterisk ($\ast$),
and similarly use an asterisk when referring to the charged lepton and neutrino originating from the decay of this $\PW$ boson.
When evaluating the integrand in Eq.~(\ref{eq:mem3}) for the signal hypothesis, 
a sizeable contribution to the value of the integral $w_{0}(\vecy|\vecyhat)$ is obtained only for those points $\vecphat$ in the $6$-particle PS for which
the mass of the system of $\Pbottom$ plus $\APbottom$ quark equals $m_{\PHiggs} = 125.1$~\GeV~\cite{HIG-14-042},
the mass of the system of charged lepton plus neutrino that originates from the decay of the on-shell $\PW$ boson equals $m_{\PW} = 80.4$~\GeV~\cite{PDG},
and the mass of the system of two charged leptons plus the two neutrinos equals $m_{\PHiggs}$.

We formally introduce these mass constraints by inserting three $\delta$-functions $\delta\left( g(x) \right)$ into the integrand of Eq.~(\ref{eq:mem3}).
The procedure is explained in Section~\ref{sec:appendix_mass_constraints} of the appendix.
More specifically, we insert
one $\delta$-function of the type $g(\Ehat_{\APbottom})$ given by Eq.~(\ref{eq:bEn_Hbb1}), 
one of the type $g(\Ehat_{\Pnu})$ given by Eq.~(\ref{eq:nuEn_Wlnu1}), and one of the type $g(\Ehat_{\nuStar})$ given by Eq.~(\ref{eq:nuEn_Hww1}).
We denote the charged leptons and neutrinos in signal events by the labels $\Plepton$ and $\Pnu$ ($\ellStar$ and $\nuStar$)
in order to indicate which charged lepton and neutrino originates from the decay of the on-shell (off-shell) $\PW$ boson and is therefore subject (not subject) to the $\PW$ mass constraint.

After solving for the $\delta$-functions analytically, as detailed in Sections~\ref{sec:appendix_bEn_Hbb}, ~\ref{sec:appendix_nuEn_Wlnu}, and~\ref{sec:appendix_nuEn_Hww} of the appendix,
the resulting expression for the probability density $w_{0}(\vecy|\vecyhat)$ for the signal hypothesis reads:
\begin{align}
w_{0}(\vecy|\vecyhat) 
 = & \frac{(m_{\PHiggs} \, \Gamma_{\PHiggs})^{2} \, m_{\PW} \, \Gamma_{\PW}}{2^{21} \, \pi^{14} \, \sigma \, E_{\Plepton} \, E_{\ellStar} \, E_{\Pbottom} \, E_{\APbottom}} \, \int \,
d\Ehat_{\Pbottom} \, d\thetahat_{\Pnu} \, d\phihat_{\Pnu} \, d\thetahat_{\nuStar} \, d\phihat_{\nuStar}  \nonumber \\
 & \quad \cdot \, \underbrace{\betahat_{\Pnu}}_{= 1} \, \underbrace{\Ehat_{\Pnu} \, \sin\thetahat_{\Pnu}}_{= \pThat^{\Pnu}} \, 
  \underbrace{\betahat_{\nuStar}}_{= 1} \, \underbrace{\Ehat_{\nuStar} \, \sin\thetahat_{\nuStar}}_{= \pThat^{\nuStar}} \, 
\frac{f(\xhat_{a}) \, f(\xhat_{b})}{\xhat_{a} \, \xhat_{b} \, s} \nonumber \\
 & \quad \cdot \, \vert \mathcal{M}_{0}(\vecphat) \vert^{2} \, 
\frac{\betahat_{\Pbottom} \, \Ehat_{\Pbottom}}{\beta_{\Pbottom} \, E_{\Pbottom}} \, W(E_{\Pbottom}|\Ehat_{\Pbottom}) \, 
\frac{\betahat_{\APbottom} \, \Ehat_{\APbottom}}{\beta_{\APbottom} \, E_{\APbottom}} \, W(E_{\APbottom}|\Ehat_{\APbottom}) \,
W_{\rho}( \pX^{\rho},\pY^{\rho} | \pXhat^{\rho},\pYhat^{\rho} ) \nonumber \\
 & \quad \cdot \, \frac{1}{\Ehat_{\Pbottom} \, \left( 1 - \frac{\betahat_{\Pbottom}}{\betahat_{\APbottom}} \, \cos\sphericalangle(\vecehat_{\Pbottom},\vecehat_{\APbottom}) \right)} \nonumber \\
 & \quad \cdot \, \frac{1}{4 \, \Ehat_{\Plepton} \, \sin^{2}\left(\frac{\sphericalangle(\vecehat_{\Plepton},\vecehat_{\Pnu})}{2}\right)} \,
\frac{1}{\Ehat_{\ellnuellStar} \left( 1 - \betahat_{\ellnuellStar} \, \cos\sphericalangle(\vecehat_{\ellnuellStar},\vecehat_{\nuStar}) \right)} \nonumber \\
 = & \frac{(m_{\PHiggs} \, \Gamma_{\PHiggs})^{2} \, m_{\PW} \, \Gamma_{\PW}}{2^{23} \, \pi^{14} \, \sigma \, s \, 
  E_{\Plepton} \, E_{\ellStar} \, \beta_{\Pbottom} \, E_{\Pbottom}^{2} \, \beta_{\APbottom} \, E_{\APbottom}^{2}} \, \int \,
d\Ehat_{\Pbottom} \, d\thetahat_{\Pnu} \, d\phihat_{\Pnu} \, d\thetahat_{\nuStar} \, d\phihat_{\nuStar} \nonumber \\
 & \quad \cdot \, \pThat^{\Pnu} \, \pThat^{\nuStar} \, 
\frac{f(\xhat_{a}) \, f(\xhat_{b})}{\xhat_{a} \, \xhat_{b}} \nonumber \\
 & \quad \cdot \, \vert \mathcal{M}_{0}(\vecphat) \vert^{2} \, 
\betahat_{\Pbottom} \, \Ehat_{\Pbottom} \, W(E_{\Pbottom}|\Ehat_{\Pbottom}) \, 
\betahat_{\APbottom} \, \Ehat_{\APbottom} \, W(E_{\APbottom}|\Ehat_{\APbottom}) \,
W_{\rho}( \pX^{\rho},\pY^{\rho} | \pXhat^{\rho},\pYhat^{\rho} ) \nonumber \\
 & \quad \cdot \, \frac{1}{\Ehat_{\Pbottom} \, \left( 1 - \frac{\betahat_{\Pbottom}}{\betahat_{\APbottom}} \, \cos\sphericalangle(\vecehat_{\Pbottom},\vecehat_{\APbottom}) \right)} \nonumber \\
 & \quad \cdot \, \frac{1}{\Ehat_{\Plepton} \, \sin^{2}\left(\frac{\sphericalangle(\vecehat_{\Plepton},\vecehat_{\Pnu})}{2}\right)} \,
\frac{1}{\Ehat_{\ellnuellStar} \left( 1 - \betahat_{\ellnuellStar} \, \cos\sphericalangle(\vecehat_{\ellnuellStar},\vecehat_{\nuStar}) \right)} \, ,
\label{eq:mem_signal}
\end{align}
with:
\begin{eqnarray}
\Ehat_{\APbottom} & = & \frac{a \, \Delta_{m_{\PHiggs}} + |b| \, \sqrt{\Delta_{m_{\PHiggs}}^{2} - (a^{2} - b^{2}) \, m_{\Pbottom}^{2}}}{a^{2} - b^{2}} \nonumber \\
\Ehat_{\Pnu} & = & \frac{m_{\PW}^{2}}{4 \, \Ehat_{\Plepton} \, \sin^{2}\left(\frac{\sphericalangle(\vecehat_{\Plepton},\vecehat_{\Pnu})}{2}\right)} \nonumber \\
\Ehat_{\nuStar} & = & \frac{m_{\PHiggs}^{2} - m_{\ellnuellStar}^{2}}{2 \, \Ehat_{\ellnuellStar} \, 
 \left( 1 - \betahat_{\ellnuellStar} \, \cos\sphericalangle(\vecehat_{\ellnuellStar},\vecehat_{\nuStar}) \right)} \, ,
\end{eqnarray}
where:
\begin{eqnarray}
\Delta_{m_{\PHiggs}} & = & \frac{m_{\PHiggs}^{2}}{2} - m_{\Pbottom}^{2} \nonumber \\
a & = & \Ehat_{\Pbottom} \nonumber \\
b & = & \underbrace{\sqrt{\Ehat_{\Pbottom}^{2} - m_{\Pbottom}^{2}}}_{= \betahat_{\Pbottom} \, \Ehat_{\Pbottom}} \, 
 \underbrace{\vecehat_{\Pbottom} \cdot \vecehat_{\APbottom}}_{= \cos\sphericalangle(\vecehat_{\Pbottom},\vecehat_{\APbottom})} \, 
= \, \betahat_{\Pbottom} \, \Ehat_{\Pbottom} \, \cos\sphericalangle(\vecehat_{\Pbottom},\vecehat_{\APbottom}) \, .
\end{eqnarray}
The integral given by Eq.~(\ref{eq:mem_signal}) extends over $5$ remaining variables,
which are integrated numerically: $\Ehat_{\Pbottom}$, $\thetahat_{\Pnu}$, $\phihat_{\Pnu}$, $\thetahat_{\nuStar}$, and $\phihat_{\nuStar}$.

There is one further aspect that needs to be taken into account when evaluating the compatibility of a given event with the signal hypothesis,
and that is that there exists a fourfold ambiguity in associating the two measured $\Pbottom$-jets to the $\Pbottom$ and $\APbottom$ quarks 
and in associating the two measured charged leptons to the on-shell and off-shell $\PW$ bosons.
We deal with the fourfold ambiguity by evaluating the integral $w_{0}(\vecy|\vecyhat)$ given by Eq.~(\ref{eq:mem_signal}) four times,
each time permuting the association of the measured $\Pbottom$-jets to the $\Pbottom$ and $\APbottom$ quarks and of the measured charged leptons to the on-shell and off-shell $\PW$ bosons.


\subsection{Analytic transformations specific to the background hypothesis}
\label{sec:mem_background}

In $\ttbar \to \Pbottom\PW^{+} \, \APbottom\PW^{-} \to \Pbottom\ellPlusnu \, \APbottom\ellMinusnu$ background events,
both $\PW$ bosons are on-shell and sizeable contributions to the value of the integral $w_{1}(\vecy|\vecyhat)$ are thus obtained only
for those points $\vecphat$ in the $6$-particle PS for which
the masses of the $\ellPlusnu$ as well as of the $\ellMinusnu$ system are equal to $m_{\PW} = 80.4$~\GeV~\cite{PDG},
and the masses of the $\Pbottom\ellPlusnu$ and $\APbottom\ellMinusnu$ systems are equal to the top quark mass of $m_{\Ptop} = 173.0$~\GeV~\cite{PDG}.

We account for these mass constraints by inserting four $\delta$-functions $\delta\left( g(x) \right)$ into the integrand of Eq.~(\ref{eq:mem3}):
two $\delta$-functions of the type $g(\Ehat_{\Pnu})$ given by Eq.~(\ref{eq:nuEn_Wlnu1}) 
and two $\delta$-functions of the type $g(\Ehat_{\Pbottom})$ given by Eq.~(\ref{eq:bEn_top1}).
Note that the first (second) $\delta$-function of each type refers to the decay of the top (anti-top) quark and hence carries the labels $\Pnu$ and $\Pbottom$ 
(and hence the labels $\Pnu$ and $\Pbottom$ need to be replaced by new labels $\APnu$ and $\APbottom$).

After solving for the $\delta$-functions analytically, following Sections~\ref{sec:appendix_nuEn_Wlnu} and~\ref{sec:appendix_bEn_top} of the appendix,
the resulting expression for the integral $w_{1}(\vecy|\vecyhat)$ for the background hypothesis is given by:
\begin{align}
w_{1}(\vecy|\vecyhat) 
 = & \frac{(m_{\Ptop} \, \Gamma_{\Ptop})^{2} \, (m_{\PW} \, \Gamma_{\PW})^{2}}{2^{21} \, \pi^{14} \, \sigma \, E_{\Plepton^{+}} \, E_{\Plepton^{-}} \, E_{\Pbottom} \, E_{\APbottom}} \, \int \,
d\thetahat_{\Pnu} \, d\phihat_{\Pnu} \, d\thetahat_{\APnu} \, d\phihat_{\APnu}  \nonumber \\
 & \quad \cdot \, \underbrace{\betahat_{\Pnu}}_{= 1} \, \underbrace{\Ehat_{\Pnu} \, \sin\thetahat_{\Pnu}}_{= \pThat^{\Pnu}} \, 
  \underbrace{\betahat_{\APnu}}_{= 1} \, \underbrace{\Ehat_{\APnu} \, \sin\thetahat_{\APnu}}_{= \pThat^{\APnu}} \, 
\frac{f(\xhat_{a}) \, f(\xhat_{b})}{\xhat_{a} \, \xhat_{b} \, s} \nonumber \\
 & \quad \cdot \, \vert \mathcal{M}_{1}(\vecphat) \vert^{2} \, 
\frac{\betahat_{\Pbottom} \, \Ehat_{\Pbottom}}{\beta_{\Pbottom} \, E_{\Pbottom}} \, W(E_{\Pbottom}|\Ehat_{\Pbottom}) \, 
\frac{\betahat_{\APbottom} \, \Ehat_{\APbottom}}{\beta_{\APbottom} \, E_{\APbottom}} \, W(E_{\APbottom}|\Ehat_{\APbottom}) \,
W_{\rho}( \pX^{\rho},\pY^{\rho} | \pXhat^{\rho},\pYhat^{\rho} ) \nonumber \\
 & \quad \cdot \, \frac{1}{4 \, \Ehat_{\Plepton^{+}} \, \sin^{2}\left(\frac{\sphericalangle(\vecehat_{\ellPlus},\vecehat_{\Pnu})}{2}\right)} \, 
\frac{1}{4 \, \Ehat_{\Plepton^{-}} \, \sin^{2}\left(\frac{\sphericalangle(\vecehat_{\ellMinus},\vecehat_{\APnu})}{2}\right)} \nonumber \\
 & \quad \cdot \, \frac{1}{\Ehat_{\ellPlusnu} \, \left( 1 - \frac{\betahat_{\ellPlusnu}}{\betahat_{\Pbottom}} \, \cos\sphericalangle(\vecehat_{\ellPlusnu},\vecehat_{\Pbottom}) \right)} \,
\frac{1}{\Ehat_{\ellMinusnu} \, \left( 1 - \frac{\betahat_{\ellMinusnu}}{\betahat_{\APbottom}} \, \cos\sphericalangle(\vecehat_{\ellMinusnu},\vecehat_{\APbottom}) \right)} \nonumber \\
 = & \frac{(m_{\Ptop} \, \Gamma_{\Ptop})^{2} \, (m_{\PW} \, \Gamma_{\PW})^{2}}{2^{25} \, \pi^{14} \, \sigma \, s \, 
  E_{\Plepton^{+}} \, E_{\Plepton^{-}} \, \beta_{\Pbottom} \, E_{\Pbottom}^{2} \, \beta_{\APbottom} \, E_{\APbottom}^{2}} \, \int \,
d\thetahat_{\Pnu} \, d\phihat_{\Pnu} \, d\thetahat_{\APnu} \, d\phihat_{\APnu}  \nonumber \\
 & \quad \cdot \, \pThat^{\Pnu} \, \pThat^{\APnu} \,
\frac{f(\xhat_{a}) \, f(\xhat_{b})}{\xhat_{a} \, \xhat_{b}} \nonumber \\
 & \quad \cdot \, \vert \mathcal{M}_{1}(\vecphat) \vert^{2} \, 
\betahat_{\Pbottom} \, \Ehat_{\Pbottom} \, W(E_{\Pbottom}|\Ehat_{\Pbottom}) \, 
\betahat_{\APbottom} \, \Ehat_{\APbottom} \, W(E_{\APbottom}|\Ehat_{\APbottom}) \,
W_{\rho}( \pX^{\rho},\pY^{\rho} | \pXhat^{\rho},\pYhat^{\rho} ) \nonumber \\
 & \quad \cdot \, \frac{1}{\Ehat_{\Plepton^{+}} \, \sin^{2}\left(\frac{\sphericalangle(\vecehat_{\ellPlus},\vecehat_{\Pnu})}{2}\right)} \, 
\frac{1}{\Ehat_{\Plepton^{-}} \, \sin^{2}\left(\frac{\sphericalangle(\vecehat_{\ellMinus},\vecehat_{\APnu})}{2}\right)} \nonumber \\
 & \quad \cdot \, \frac{1}{\Ehat_{\ellPlusnu} \, \left( 1 - \frac{\betahat_{\ellPlusnu}}{\betahat_{\Pbottom}} \, \cos\sphericalangle(\vecehat_{\ellPlusnu},\vecehat_{\Pbottom}) \right)} \,
\frac{1}{\Ehat_{\ellMinusnu} \, \left( 1 - \frac{\betahat_{\ellMinusnu}}{\betahat_{\APbottom}} \, \cos\sphericalangle(\vecehat_{\ellMinusnu},\vecehat_{\APbottom}) \right)} \, ,
\label{eq:mem_background}
\end{align}
with:
\begin{eqnarray}
\Ehat_{\Pbottom} & = & \frac{a_{\Ptop} \, \Delta_{m_{\Ptop}}
 + |b_{\Ptop}| \, \sqrt{\Delta_{m_{\Ptop}}^{2} - (a_{\Ptop}^{2} - b_{\Ptop}^{2}) \, m_{\Pbottom}^{2}}}{a_{\Ptop}^{2} - b_{\Ptop}^{2}} \nonumber \\
\Ehat_{\APbottom} & = & \frac{a_{\APtop} \, \Delta_{m_{\Ptop}}
 + |b_{\APtop}| \, \sqrt{\Delta_{m_{\Ptop}}^{2} - (a_{\APtop}^{2} - b_{\APtop}^{2}) \, m_{\Pbottom}^{2}}}{a_{\APtop}^{2} - b_{\APtop}^{2}} \nonumber \\
\Ehat_{\Pnu} & = & \frac{m_{\PW}^{2}}{4 \, \Ehat_{\ellPlus} \, \sin^{2}\left(\frac{\sphericalangle(\vecehat_{\ellPlus},\vecehat_{\Pnu})}{2}\right)} \nonumber \\
\Ehat_{\APnu} & = & \frac{m_{\PW}^{2}}{4 \, \Ehat_{\ellMinus} \, \sin^{2}\left(\frac{\sphericalangle(\vecehat_{\ellMinus},\vecehat_{\APnu})}{2}\right)} \, ,
\end{eqnarray}
where:
\begin{eqnarray}
\Delta_{m_{\Ptop}} & = & \frac{m_{\Ptop}^{2} - m_{\Pbottom}^{2} - m_{\PW}^{2}}{2} \nonumber \\
a_{\Ptop} & = & \Ehat_{\ellPlusnu} \nonumber \\
b_{\Ptop} & = & \sqrt{\Ehat_{\ellPlusnu}^{2} - m_{\PW}^{2}} \, 
 \underbrace{\vecehat_{\ellPlusnu} \cdot \vecehat_{\Pbottom}}_{= \cos\sphericalangle(\vecehat_{\ellPlusnu},\vecehat_{\Pbottom})} \, 
= \, \sqrt{\Ehat_{\ellPlusnu}^{2} - m_{\PW}^{2}} \, \cos\sphericalangle(\vecehat_{\ellPlusnu},\vecehat_{\Pbottom}) \nonumber \\
a_{\APtop} & = & \Ehat_{\ellMinusnu} \nonumber \\
b_{\APtop} & = & \sqrt{\Ehat_{\ellMinusnu}^{2} - m_{\PW}^{2}} \, 
 \underbrace{\vecehat_{\ellMinusnu} \cdot \vecehat_{\APbottom}}_{= \cos\sphericalangle(\vecehat_{\ellMinusnu},\vecehat_{\APbottom})} \,
= \, \sqrt{\Ehat_{\ellMinusnu}^{2} - m_{\PW}^{2}} \, \cos\sphericalangle(\vecehat_{\ellMinusnu},\vecehat_{\APbottom}) \, .
\end{eqnarray}
The integral given by Eq.~(\ref{eq:mem_background}) extends over $4$ remaining variables,
which are integrated numerically: $\thetahat_{\Pnu}$, $\phihat_{\Pnu}$, $\thetahat_{\APnu}$, and $\phihat_{\APnu}$.

When evaluating the compatibility of a given event with the background hypothesis,
there exists a twofold ambiguity in associating the two measured $\Pbottom$-jets to the $\Pbottom$ and $\APbottom$ quarks.
We deal with this ambiguity by evaluating the integral $w_{1}(\vecy|\vecyhat)$ given by Eq.~(\ref{eq:mem_signal}) two times,
each time permuting the association of the measured $\Pbottom$-jets to the $\Pbottom$ and $\APbottom$ quarks.
Note that in case of the background hypothesis, there is no ambiguity in associating the two measured leptons to the $\PW^{+}$ and $\PW^{-}$ bosons,
as both $\PW$ bosons are on-shell and the measurement of the lepton charge allows for a unique association of each charged lepton to either the $\PW^{+}$ or the $\PW^{-}$ boson,
as conservation of electric charge in the decays $\PW^{+} \to \ellPlusnu$ and $\PW^{-} \to \ellMinusnu$ implies that the the charge of each lepton $\Plepton$ matches the charge of the $\PW$ boson
from the decay of which it originates.
