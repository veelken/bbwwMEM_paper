\section{Extension to BSM  kinematics}

So far, we have focused on non-resonant $\dihiggs$ production with SM-like event kinematics. 
The $\dihiggs$ production rate may be enhanced significantly in case an as yet unknown resonance decays to pairs of $\PHiggs$ bosons.
Such resonances are predicted in models with two Higgs doublets~\cite{Branco:2011iw}, composite $\PHiggs$ boson models~\cite{Grober:2010yv}, 
Higgs portal models~\cite{Patt:2006fw}, and models involving extra dimensions~\cite{Randall:1999ee}. 
In the absence of new resonances decaying into $\PHiggs$ boson pairs,
the $\dihiggs$ production cross section may be enhanced by the contribution of BSM interactions and/or new particles to the loops 
that are present in the triangle and box diagrams shown in Fig.~\ref{fig:ggHH_FeynmanDiagram}. 
The effect of such  can adequately be described by anomalous $\PHiggs$ boson couplings. 
The production of $\PHiggs$ boson pairs via gluon fusion in Beyond the Standard Model (BSM) theories can be described to leading approximation by the non-linear  Lagrangian~\cite{Buchmuller:1985jz}:
\begin{eqnarray}
\mathcal{L} & = & \frac{1}{2} \partial_{\mu}\PHiggs\partial^{\mu}\PHiggs - \frac{1}{2} m_{\PHiggs}^{2} \PHiggs^{2} - \kappa_{\lambda} \lambdaHHH^{\textrm{SM}} v \PHiggs^{3} \nonumber \\
 & & \quad -\frac{m_{\Ptop}}{v} \, \left( v + \kappa_{\Ptop} \PHiggs + \frac{c_{2}}{v} \PHiggs\PHiggs \right) \, \left( \bar{\Ptop}_{\textrm{L}} \Ptop_{\textrm{R}} + \textrm{h.c.} \right) 
+ \frac{1}{4} \frac{\alpha_{\textrm{s}}}{3 \pi v} \, \left( c_{\Pgluon} \PHiggs - \frac{c_{2\Pgluon}}{2 v} \PHiggs\PHiggs \right) \, \textrm{G}^{\mu\nu}\textrm{G}_{\mu\nu} \, ,
\end{eqnarray}
where $v = 246$~\GeV denotes the vacuum expectation value of the Higgs field
and the parameters $\kappa_{\lambda} = \lambdaHHH/\lambdaHHH^{\textrm{SM}}$ and $\kappa_{\Ptop} = \yt/\yt^{\textrm{SM}}$ 
quantify the deviations of the trilinear Higgs self-coupling and of the coupling of the $\PHiggs$ boson to the top quark from their SM values.
Besides the two parameters $\kappa_{\lambda}$ and $\kappa_{\Ptop}$,
the Lagrangian contains the coefficients of three BSM operators which account for contact interactions 
between a $\PHiggs$ boson and either one ($c_{\Pgluon}$) or two ($c_{2\Pgluon}$) gluons
and between a pair of $\PHiggs$ bosons and a pair of top quarks ($c_{2}$)~\footnote{ \textcolor{red}{X: explain why we ignore the chomomagnetic here}}.

These couplings can be related with an effective field theory (EFT) approach
In the context of the EFT approach~\cite{Buchalla:2015wfa, Goertz:2014qta} or as the connection to a ultraviolet theory~\cite{Carvalho:2017vnu,Belusca-Maito:2016dqe,deBlas:2014mba,Dawson:2017vgm}. %The production of $\PHiggs$ boson pairs in the absence of new resonances is referred to as non-resonant $\dihiggs$ production.
%We focus on the case of non-resonant $\dihiggs$ production in this paper.
The distribution in mass of the $\PHiggs$ boson pair, $m_{\dihiggs}$, is rather broad in this case.
Deviations of  the $\PHiggs$ boson couplings from their SM values, 
 induces different interference patterns between the different production diagrams and may  significantly alter the distribution in $m_{\dihiggs}$,
as well as the rest of the event kinematics,
as the distributions in the momenta and angles of the particles reconstructed in the final state,
besides affecting the $\dihiggs$ production cross section.


In case the $\dihiggs$ production rate is enhanced compared to the SM cross section, the scenario probed by the present searches for $\dihiggs$ production at the LHC,
 contributions of BSM physics also alter the event kinematics. To simplify the classification of the possible shape modifications Ref.~\cite{Carvalho:2015ttv} had proposed the definition of the $12$ shape benchmark scenarios. We will investigate the performance of the MEM on a BSM-like signal by computing the likelihood ratios $P$ and $P_{\textrm{m}}$ it to each one of the shape-benchmark points. \textcolor{red}{X: Stoped writing here.}  
 
 \begin{figure}
\setlength{\unitlength}{1mm}
\begin{center}
BLABLA
\end{center}
\caption{
  MEM for the shape benchmarks: maybe just the roc, summarizing all?
}
\label{fig:ttbar_FeynmanDiagram}
\end{figure}
 
 
In the context of an EFT approach to probe the existence of such BSM physics we then suggest follow the approach suggested on~\cite{Carvalho:2016rys}:  run a parametrized training of the BDT or NN for these shape benchmark scenarios. \textcolor{red}{If we want to explain that I need to spend some time explaining the method of finding the closest  shape benchmark.... it goes as other 1/2 page paragraph -- the sentence bellow makes no sense, and I commented we already said that the model allows parameters.} 
%\textcolor{blue}{Using the same model [cite] we can also }  ME $\vert \mathcal{M}_{0} \vert^{2}$ \st{obtained from the program MadGraph\_aMCatNLO 2.2.2~\cite{MadGraph_aMCatNLO}[CV: CHECK MADGRAPH VERSION USED TO GENERATE ME]}
%allows to set the two parameters $\kappa_{\lambda}$ and $\kappa_{\Ptop}$ and the coefficients $c_{2}$, $c_{\Pgluon}$, and $c_{2\Pgluon}$
%to the values given in Table~1 of Ref.~\cite{Carvalho:2015ttv} for the SM case and each of the $12$ benchmark scenarios.


