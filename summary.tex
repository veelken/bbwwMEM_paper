\section{Summary}
\label{sec:summary}

We presented an application of the matrix element method 
to the search for non-resonant $\dihiggs$ production in the channel $\dihiggs \to \Pbottom\Pbottom\PW\PW^{*}$ at LHC,
in events in which the two $\PW$ bosons decay to a pair of electrons or muons.
The likelihood ratio $P(\vecy)$ given by Eq.~(\ref{eq:memLR}) provides the optimal separation of the $\dihiggs$ signal from the dominant irreducible $\ttbar$ background
according to the Neyman–Pearson lemma.
A cut on $P(\vecy)$ achieves a reduction of the $\ttbar$ background by three orders of magnitude for a signal efficiency of $35\%$
in case the probability densities $w_{0}(\vecy|\vecyhat)$ and $w_{1}(\vecy|\vecyhat)$ that quantify the level of compatibility with the signal and background hypotheses
are evaluated on the true momenta of the charged leptons, $\Pbottom$-jets, and of the missing transverse momentum produced in the decay of the $\PHiggs$ boson or top quark pair. \textcolor{red}{X: Ideally? } 

We find that the experimental resolution on the energy of $\Pbottom$-jets and on the hadronic recoil has a small effect on the separation between signal and background,
whereas the signal-to-background separation degrades significantly in case one of the two $\Pbottom$-jets that are produced in the decays of the $\PHiggs$ boson or top quark pair
fails to get reconstructed and a light quark or gluon jet gets misidentified as $\Pbottom$-jet.
Ideas to mitigate the resulting degradation in performance and to handle other, subdominant, backgrounds have been presented.
The usefulness of these ideas yet remains to be studied in a full analysis, employing detailed simulations of the ATLAS and CMS detectors, which is beyond the scope of the present paper.
