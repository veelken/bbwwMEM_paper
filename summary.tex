\section{Summary}
\label{sec:summary}

We presented an application of the matrix element method 
to the search for non-resonant $\dihiggs$ production in the channel $\dihiggs \to \Pbottom\APbottom\PW\PW\virt$ at LHC,
focusing on events in which the two $\PW$ bosons decay to a pair of electrons or muons.
According to the Neyman-Pearson lemma,
the likelihood ratio $P(\vecy)$ given by Eq.~(\ref{eq:memLR}) provides the optimal separation of the $\dihiggs$ signal from the dominant irreducible $\ttbar$ background.
We have studied the separation of the $\dihiggs$ signal from the $\ttbar$ background at Monte-Carlo truth and at detector level.
The latter has been simulated using the $\textsc{DELPHES}$ fast-simulation framework.
For experimental conditions characteristic for the ATLAS and CMS experiments during LHC Run $2$,
we find that the $\ttbar$ background can be reduced to a level of $0.26\%$ for a signal efficiency of $35\%$.
We regard the potential of the matrix element method for enhancing the sensitivity of the analysis of $\dihiggs$ production in the channel $\dihiggs \to \Pbottom\APbottom\PW\PW\virt$
as promising and we hope this paper will motivate the ATLAS and CMS collaborations to employ the method in a full analysis.
