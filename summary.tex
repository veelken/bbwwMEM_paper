\section{Summary}
\label{sec:summary}

We presented an application of the matrix element method 
to the search for non-resonant $\dihiggs$ production in the channel $\dihiggs \to \Pbottom\APbottom\PW\PW^{*}$ at LHC,
in events in which the two $\PW$ bosons decay to a pair of electrons or muons.
According to the Neyman–Pearson lemma,
the likelihood ratio $P(\vecy)$ given by Eq.~(\ref{eq:memLR}) provides the optimal separation of the $\dihiggs$ signal from the dominant irreducible $\ttbar$ background.
For a signal efficiency of $35\%$,
a reduction of the $\ttbar$ background by three orders of magnitude can be achieved in case the two jets corresponding to the $\Pbottom$ and $\APbottom$ quarks are identified as $\Pbottom$-jets
and the momenta of charged leptons, the energy of $\Pbottom$-jets, and the components $\pX$ and $\pY$ of the hadronic recoil (or, equivalently, of the missing transverse momentum)
are measured with negligible experimental resolution.

We find that the experimental resolution on the energy of $\Pbottom$-jets and on the hadronic recoil has only a small effect on the separation between signal and background,
whereas the signal-to-background separation degrades significantly in case one of the two $\Pbottom$-jets
fails to get reconstructed and a light quark or gluon jet gets misidentified as $\Pbottom$-jet instead.
Ideas to mitigate the resulting degradation in performance and to handle other, subdominant, backgrounds have been presented.
The usefulness of these ideas yet remains to be studied in a full analysis, employing detailed simulations of the ATLAS and CMS detectors, which is beyond the scope of the present paper.
